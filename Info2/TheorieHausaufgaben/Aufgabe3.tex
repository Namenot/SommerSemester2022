\documentclass[12pt, a4paper, oneside]{article}

\usepackage[utf8]{inputenc}
\usepackage{tikz}


\usepackage[
a4paper,
textwidth=175mm,
textheight=225mm,
heightrounded,
]{geometry}

\usepackage[T1]{fontenc} % <-- don't forget
\usepackage[utf8]{inputenc}
\usepackage[dutch]{babel}
\usepackage{graphicx}
\usepackage{pgfplots}
\usepackage[
backend=biber,
style=alphabetic,
sorting=ynt,
]{biblatex}
\addbibresource{references.bib}

\usepackage[hidelinks]{hyperref}

\usepackage[
a4paper,
textwidth=175mm,
textheight=225mm,
heightrounded,
]{geometry}

\usepackage[T1]{fontenc} % <-- don't forget
\usepackage[utf8]{inputenc}
\usepackage[dutch]{babel}
\usepackage{graphicx}
\usepackage{pgfplots}
\usepackage[
backend=biber,
style=alphabetic,
sorting=ynt,
]{biblatex}
\addbibresource{references.bib}

\usepackage[hidelinks]{hyperref}

\setlength{\parskip}{1.2ex}    % space between paragraphs
\setlength{\parindent}{0cm}%2em}       % amount of indention

\title{Hausaufgabe 3}
\author{Aaron Sastry}

\begin{document}
	
	\maketitle
	
	\textbf{Aufgabe 3.1 Radix-Sort}
	\begin{description}
	
	\item [a)] \hfill \\
	Sortieren Sie die Folge a = (271, 842, 172, 828, 904, 023, 991, 800, 601, 889) mit LSD–
	RadixSort
	
	\emph{1st 10 buckets}
	\begin{tabular}{| c| c| c| c| c| c| c| c| c| c|}
		
		\hline
		B0 & B1 & B2 & B3 & B4 & B5 & B6 & B7 & B8 & B9 \\
		\hline\hline
		
		800 & 271 & 842 & 023 & 904 & & & & 828 & 889  \\
		    & 991 & 172 & & & & & & &  \\
		    & 601 & & & & & & & & \\ 
		\hline
	\end{tabular}

	output 1: (800, 271, 991, 601, 842, 172, 023, 904, 828, 889)
	\newline
	
	\emph{2nd 10 buckets}
	\begin{tabular}{| c| c| c| c| c| c| c| c| c| c|}
		
		\hline
		B0 & B1 & B2 & B3 & B4 & B5 & B6 & B7 & B8 & B9 \\
		\hline\hline
		
		800 & & 023 & & 842 & & & 271 & 889 & 991 \\
		601 & & 828 & &     & & & 172 & & \\
		904 & & & &     & & &     & & \\
		\hline
	\end{tabular}
	
	output 2: (800, 601, 904, 023, 828, 842, 271, 172, 889, 991)
	\newline
	
	\emph{3rd 10 buckets}
	\begin{tabular}{| c| c| c| c| c| c| c| c| c| c|}
		
		\hline
		B0 & B1 & B2 & B3 & B4 & B5 & B6 & B7 & B8 & B9 \\
		\hline\hline
		
		023 & 172 & 271 & & & & 601 & & 800 & 904 \\
		    &     &     & & & &     & & 828 & 991 \\
		    &     &     & & & &     & & 842 &     \\
		    &     &     & & & &     & & 889 &     \\
		
		\hline
	\end{tabular}
	
	output 3: (023, 172, 271, 601, 800, 828, 842, 889, 904, 991)
	\newline
	\emph{Nach 3 Iterationen ist die List final sortiert mit der Reihenfolge:}
	\newline
	\textbf{\emph{023, 172, 271, 601, 800, 828, 842, 889, 904, 991}}
	
	\newpage
	\item[b)] \hfill \\
	Sortieren Sie die Folge b = (0.271, 0.842, 0.172, 0.828, 0.904) mit MSD–RadixSort.
	\\ \\
	
	\emph{1st set of buckets}\\
	\begin{tabular}{| c| c| c| c| c| c| c| c| c| c|}
		
		\hline
		B0 & B1 & B2 & B3 & B4 & B5 & B6 & B7 & B8 & B9 \\
		\hline\hline
		
		& 0.172 & 0.271 & & & & & & 0.842 & 0.904 \\
		&       &       & & & & & & 0.828 &  \\
		
		\hline
	\end{tabular}

	output: 1 = ([0.172], [0.271], [0.842, 0.828], [0.904]) \\
	
	Jetzt werden nur noch die Buckets sortiert die mehr als 1 Element haben.
	
	\emph{2nd set of buckets, }
	\emph{(the 0.8 bucket)}\\
	\begin{tabular}{| c| c| c| c| c| c| c| c| c| c|}
		
		\hline
		B0 & B1 & B2 & B3 & B4 & B5 & B6 & B7 & B8 & B9 \\
		\hline\hline
		
		& & 0.828 & & 0.842 & & & & & \\
		
		\hline
	\end{tabular}
	
	output: ([0.172], [0.271], [[0.828], [0.842]], [0.904]) \\
	
	Zu guter Letzt werden jetzt noch die Buckets wieder zu einer fertigen Liste zusammen gesetzt:
	
	\textbf{$\Longrightarrow$ \emph{0.172, 0.271, 0.828, 0.842, 0.904}}
	\newline
	
	\item[c)] \hfill \\
	Angenommen, Sie wollen m-stellige Binärzahlen mit MSD-RadixSort sortieren, wobei
	die Behälter rekursiv wieder mit MSD-RadixSort sortiert werden sollen. Was ist die
	Laufzeit dieses Sortieralgorithmus für n Binärzahlen, abhängig von n und m? Begründen
	Sie Ihre Behauptung kurz.
	\newline
	
	
	\textbf{Antwort:}
	\newline
	Aus jeder liste entstehen immmer 2 buckets auf denen wieder sortiert werden muss.
	Somit sind in der iteration i etwa : \(2^i\)  befüllte Buckets.
	Da alle Buckets vereinigt wieder genau n Elemente haben ist jede Iteration in
	\begin{center}
		\(T = n\)
	\end{center} 
	Nun gibt es natürlich etwa \(2^m\) Buckets und exakt \emph{m} operationen da dies die Anzahl der stellen sind in die wir einteilen.
	Somit erfolgt die Einteilung in die gesammten Buckets in
	\begin{center}
		 \(T = m*n\)
	\end{center}
	Zu guter Letzt fehlt noch die vereinigung der Buckets wieder in eine vollsändige List, was wieder genau \(T = n\)
	braucht. 
	Somit ist die finale Laufzeit bei:
	\begin{center}
		\item\(T = (m+1)*n \) \(\Longleftrightarrow T = m*n + n\)
		\item \emph{\( \Longrightarrow \mathcal{O}(m*n) \)}\\
	\end{center}
	\end{description}
	
\end{document}